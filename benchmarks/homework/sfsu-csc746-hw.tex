% 5/22/2025 updated to use more common IEEEtran class file
% see https://www.ieee.org/conferences/publishing/templates
\documentclass[11pt,final,conference]{IEEEtran}

\usepackage{cite}
\usepackage{physics,amsmath,amssymb,amsfonts}
\usepackage{algorithmic}
\usepackage{graphicx}
\usepackage{textcomp}
\usepackage{xcolor}


%% wes 8/2021 additions
\usepackage{comment}    % wes 8/2021
\usepackage{color}      % wes 8/2021
\usepackage{listings}   % wes 8/2021

% wes: for code formatting/coloring
\definecolor{codegreen}{rgb}{0,0.6,0}
\definecolor{codegray}{rgb}{0.5,0.5,0.5}
\definecolor{codepurple}{rgb}{0.58,0,0.82}
\definecolor{backcolour}{rgb}{0.95,0.95,0.92}
\definecolor{codecyan}{rgb}{0.0,0.2,1.0}

% see: https://en.wikibooks.org/wiki/LaTeX/Source_Code_Listings
% set font, size, color style for code listings
\lstdefinestyle{mystyle}{
%    backgroundcolor=\color{backcolour},   
    commentstyle=\textcolor{codegreen},
%    keywordstyle=\color{magenta},    
    keywordstyle=\color{codecyan},
    numberstyle=\tiny\color{codegray},
    stringstyle=\color{codepurple},
    basicstyle=\ttfamily\footnotesize,
    breakatwhitespace=false,    
    breaklines=true,    
    captionpos=b,    
    keepspaces=true,    
    numbers=left,    
    numbersep=2pt,  
    firstnumber=auto,
    numberblanklines=false,
    showspaces=false,
    showstringspaces=false,
    showtabs=false,
    tabsize=2
}
% and then set mystyle to be the default when doing code listings
\lstset{style=mystyle}


\begin{document}

% these force page numbering to be turned on
\thispagestyle{plain}
\pagestyle{plain}


%% Paper title.

\title{Your Title Goes Here \\Assignment \#X, CSC 746, Fall 2025}

% IEEEtran way of specifying authors
\author{
\IEEEauthorblockN{Your Name Here}
\IEEEauthorblockA{Computer Science Department\\
San Francisco State University\\
San Francisco, CA, USA\\
Email: whodis@sfsu.edu}
% \and
% \IEEEauthorblockN{Homer Simpson}
% \IEEEauthorblockA{Twentieth Century Fox\\
% Springfield, USA\\
% Email: homer@thesimpsons.com}
}

\maketitle

%% Abstract section.
\begin{abstract}
Please take a few moments and try to compose an abstract for your homework writeup. It should contain these ideas: what was the problem being studied, what was the approach (what did you implement), what are the results.
The abstract should describe the basic message of the paper, including: the problem, why your solution should be of interest, some notion that your solution is effective, and a teaser about how it has been evaluated. Cover all of this using between 75 and 150 words. Thus, the abstract is the hardest part to write. Sometimes I try to write it first, but the final version is usually composed of items drawn from the introduction, and then condensed, as the last step of writing the paper.
\end{abstract}



\input{01_introduction}
\input{02_relatedWork}
\input{03_implementation}
\input{04_evaluation}



%% if specified like this the section will be committed in review mode
\section*{Acknowledgement}
For homework writeups, you may comment out the Acknowledgements section.
The authors wish to thank A, B, C. This work was supported in part by
a grant from XYZ.
\textit{Comment:} you don't need this section in homework assignments, but you will need it when you create a final, camera-ready version of a technical paper for publication.

%\bibliographystyle{abbrv}
\bibliographystyle{abbrv-doi}
%\bibliographystyle{abbrv-doi-narrow}
%\bibliographystyle{abbrv-doi-hyperref}
%\bibliographystyle{abbrv-doi-hyperref-narrow}

% uncomment the following line if you decide to include a bibliography (references) with your homework writeup
\bibliography{template}
\end{document}
