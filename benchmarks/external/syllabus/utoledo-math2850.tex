\documentclass[11pt]{article}

\usepackage{geometry,url}
\geometry{body={7in,9in}, centering}
\pagestyle{empty}
\usepackage{multicol}
\begin{document}

\begin{center}
\textbf{\Large ELEMENTARY MULTIVARIABLE CALCULUS}\\
The University of Toledo\\ Mathematics \& Statistics Department, 
College of Natural Sciences and Mathematics\\ MATH2850-0XX, CRN XXXXX
\end{center}


---------------------------------------------------------------------------------------------------------------------------------
\vspace{-.1in}
\begin{center}
\begin{tabular}{llll}
    \textbf{Instructor:} & \footnotesize{(Insert Name)}  &   \textbf{Class Location:}  & \footnotesize{(Insert Building/Room)}\\ 
    \textbf{Email:} & \footnotesize{(Insert Email Address)} &   \textbf{Class Day/Time:} & \footnotesize{(Insert Days/Time)}\\
    \textbf{Office Hours:} & \footnotesize{(Insert Days/Time)}   &   \textbf{Credit Hours:} & 4\\
    \textbf{Office Location:} & \footnotesize{(Insert Building/Office \#)}  & & \\
    \textbf{Office Phone:} & \footnotesize{(Insert Phone Number)}   & &\\
    \textbf{Term:}  & \footnotesize{(Insert Semester/Year)} &  &\\
\end{tabular}
\end{center}
---------------------------------------------------------------------------------------------------------------------------------

%{Spring 2006}


\bigskip

\noindent {\bf COURSE DESCRIPTION}\\
Geometry of functions of several variables, partial differentiation, 
multiple integrals, vector algebra and calculus (including Theorems of 
Green, Gauss and Stokes), and applications.\\

%%%%%%%%%%%%%%%%%%%%%%%%%%%%%%%%%%%%%%%%%%%%%%%%%%%%%%%%%%%%%%%%%%%%%%%%%
%%%%%%%%%%%%%%%%%%%%%%%%%%%%%%%%%%%%%%%%%%%%%%%%%%%%%%%%%%%%%%%%%%%%%%%%%
%% Begining of UPDATED LEARNING OUTCOMES, February 23, 2016

\noindent{\bf STUDENT LEARNING OUTCOMES:}\\

Upon successful completion of this class a student should be able to:
\begin{enumerate}
\item \textbf{Lines and Planes}: Perform and apply vector operations, including the 
dot and cross product of vectors, in the plane and space.
	
\item \textbf{Vector-valued Functions:} Differentiate and integrate vector-valued functions. 
For a position vector function of time, interpret these as velocity and acceleration.

\item \textbf{Limits:} Evaluate limits and determine the continuity and differentiability of
functions of several variables.

\item \textbf{Graphs:} Describe graphs, level curves and level surfaces of functions of several
variables.

\item \textbf{Partial Derivatives:} Find partial derivatives, directional derivatives, and 
gradients and use them to solve applied problems.

\item \textbf{Tangent Planes:} Find equations of tangent planes and normal lines to 
surfaces that are given implicitly or parametrically.

\item \textbf{Chain Rule:} Use the chain rule for functions of several variables (including 
implicit differentiation).

\item \textbf{Extrema:} For functions of several variables, find critical points using first 
partials and interpret them as relative extrema/saddle points using the second partials test. 
Find absolute extrema on a closed region. Apply these techniques to optimization problems.

%\item  Use Lagrange multipliers to solve constrained optimization problems.
\item \textbf{Multiple Integrals:} Evaluate multiple integrals in appropriate coordinate 
systems such as rectangular, polar, cylindrical and spherical coordinates and apply them 
to solve problems involving volume, surface area, density, moments and centroids.

\item \textbf{Line/Surface Integrals:} Evaluate line and surface integrals. Identify when a 
line integral is independent of path and use the Fundamental Theorem of Line Integrals 
to solve applied problems. 

\item \textbf{Vector Fields:} Find the curl and divergence of a vector field, the work done 
on an object moving in a vector field, and the flux of a field through a surface. Use these 
ideas to solve applied problems. Identify conservative fields.

\item \textbf{Important Theorems:} Introduce and use Green's Theorem, the 
Divergence (Gauss's) Theorem and Stokes's Theorem.
\end{enumerate}

\noindent {\bf PREREQUISITES}\\ 
Minimum grade of C- in Math 1860 or Math 1840 or Math 1930. Students
who enroll in Math 2850 but have not passed either prerequisite may be 
administratively dropped from the class.\\

\noindent{\bf TEXTBOOK:} 
\textit{ Calculus -- Volume III}, OpenStax (Print ISBN-13: 978-1-938168-07-9; Digital 
ISBN-13: 978-1-947172-16-6), Senior Contributing Authors: Edwin ``Jed" Herman 
and Gilbert Strang. The ebook is available for free 
at \url{https://openstax.org/details/books/calculus-volume-3}.  \\

\medskip\noindent{\bf UNIVERSITY POLICIES:}\\

\noindent{\bf POLICY STATEMENT ON NON-DISCRIMINATION ON THE BASIS OF DISABILITY (ADA)}\\
The University is an equal opportunity educational institution. Please read The University's 
Policy Statement on Nondiscrimination on the Basis of Disability Americans with Disability 
Act Compliance.\\

\noindent{\bf ACADEMIC ACCOMMODATIONS}\\
The University of Toledo is committed to providing equal access to education for all students. 
If you have a documented disability or you believe you have a disability and would like 
information regarding academic accommodations/adjustments in this course please contact 
the Student Disability Services Office (Rocket Hall 1820; 419.530.4981; 
studentdisabilitysvs@utoledo.edu) as soon as possible for more information and/or to 
initiate the process for accessing academic accommodations. For the full policy see: 
\url{http://www.utoledo.edu/offices/student-disability-services/sam/index.html}\\

%\newpage

\medskip\noindent{\bf ACADEMIC POLICIES:}\\

\noindent{\bf STUDENT PRIVACY}\\
Federal law and university policy prohibits instructors from discussing a student's grades or 
class performance with anyone outside of university faculty/staff without the student's 
written and signed consent.  This includes parents and spouses.  For details, see the 
``Confidentiality of Student Records (FERPA)" section of the University Policy Page at 
 \url{http://www.utoledo.edu/policies/academic/undergraduate/index.html}\\

\noindent{\bf MISSED CLASS POLICY}\\
If circumstances occur in accordance with The University of Toledo Missed Class 
Policy (found at 
\url{http://www.utoledo.edu/policies/academic/undergraduate/index.html} )  result in a 
student missing a quiz, test, exam or other graded item, the student must 
contact the instructor in advance by phone, e-mail or in person, provide 
official documentation to back up his or her absence, and arrange to make up the 
missed item as soon as possible.\\

\newpage 

\noindent{\bf ACADEMIC DISHONESTY}\\ 
Any act of academic dishonesty as defined by the University of Toledo policy on academic 
dishonesty (found at \url{http://www.utoledo.edu/dl/students/dishonesty.html} will result 
in  an F in the course or an F on the item in question, subject to the determination of the 
instructor.\\

\noindent{\bf GRADING AND EVALUATION}\\
The syllabus should describe the methods of evaluation whether quizzes, exams, or 
graded assignments.  The usual procedure is to give at least two 1 hour in-class exams 
and a two hour final exam. If quizzes are not used as a portion of the grade, then three 
1 hour exams are recommended.  How each evaluation method is to count toward the 
class grade should be described and a grading scale or description of a grading procedure 
should be provided. A sample reasonable distribution for this class would be:

\vspace{.1in}

%\begin{tabular}{lr}
	\begin{tabular}{l}
	\begin{tabular}{|l|c|}
		 \hline
 		Component & points\\
		  \hline
		Homework and/or  Quizzes & 30\%\\
 	 	\hline
 		Midterm Exams &	40\%\\
 		\hline
 		Final Exam &	30\% \\
 	 	\hline
 		\end{tabular}
 	\end{tabular}

\vspace{.1in}

It should be kept in mind when scheduling quizzes and exams that the last day to add/drop 
the class is the end of the second week of classes and the last day to withdraw from the 
class is the end of the tenth week.  By these dates, students like to have some measure 
of their progress in the class.


\medskip\noindent{\bf IMPORTANT DATES}\\
The instructor reserves the right to change the content of the course material if he 
perceives a need due to postponement of class caused by inclement weather, instructor 
illness, etc., or due to the pace of the course.\\

\noindent{\bf MIDTERM EXAM:}\\
{\bf FINAL EXAM:}\\

\noindent{\bf OTHER DATES}\\
The last day to drop this course is:\\	
The last day to withdraw with a grade of ``W'' from this course is:

\medskip\noindent{\bf STUDENT SUPPORT SERVICES}\\ 
Free math tutoring on a walk-in basis is available in the Math Learning and Resources 
Center located in Rm B0200 in the lower level of Carlson Library (phone ext 2176). The 
Center operates on a walk-in basis.  MLRC hours can be found at 
\url{http://www.utoledo.edu/utlc/lec/tutoring/math.html.}\\

\medskip\noindent{\bf CLASS SCHEDULE}\\ 
Syllabus should provide a list of sections to be covered and it is advisable to give a tentative 
exam schedule.  The suggested number of periods needed for each section is listed below. 
Most instructors find the syllabus to be quite crowded, so the course needs to be well paced 
to avoid cramming too much material in at the end of the semester.  Most students will 
enroll in MATH 2860 that has MATH 2850 as a prerequisite.

\newpage
\vspace*{-.2in}
\noindent {\bf Suggested Schedule}\hfill

%%%%%%%%%%%%%%%%%%%%%%%%%%%%%%%%%%%%%%%
%%%%%%%%%%%%%%%%%%%%%%%%%%%%%%%%%%%%%%%

\noindent
\begin{tabular}{llp{5in}@{\qquad}l}
     &       &                                   & \\
Chapter & 2  & \textbf{Vectors in Space}  & (total 4 hr)\\
&2.1 & \textbf{(Op.)} Vectors in the Plane; \textit{ Lines and Planes} &  \\
&2.2   & \textbf{(Op.)} Vectors in Three Dimensions; \textit{ Lines and Planes}        &  \\
&2.3   & The Dot Product; \textit{ Lines and Planes}  & 1 \\
&2.4   & The Cross Product; \textit{ Lines and Planes}    &1\\
&2.5   & Equations of Lines and Planes in Space; \textit{ Lines and Planes}  & 1.5 \\
&2.6   & Quadric Surfaces;  \textit{Graphs}    &  0.5\\
&2.7   & \textbf{(Op.)} Cylindrical and Spherical Coordinates   & \\
&       &                                       &       \\
&		&   & \\
Chapter & 3 & \textbf{Vector-Valued Functions} & (total 2.5 hr) \\
&3.1 & Vector-Valued Functions and Space Curves; \textit{Vector-valued Functions} & 1 \\
&3.2 & Calculus of Vector-Valued Functions; \textit{Vector-valued Functions} & 1\\
&3.3 &  \textbf{(Op.)} Arc Length and Curvature; \textit{Vector-valued Functions} &  \\
&3.4  &  Motion in Space; \textit{Vector-valued Functions} &  0.5\\
&       &                                   & \\
Chapter & 4 & \textbf{Differentiation of Functions of Several Variables}  & (total 9 hr) \\
&4.1   &Functions of Several Variables; \textit{Graphs} & 0.5 \\
&4.2   & Limits and Continuity; \textit{Limits}       & 1.5 \\
&4.3   &Partial Derivatives; \textit{Partial Derivatives}        & 1\\
&4.4   & Tangent Planes and Linear Approximations; \textit{Tangent Planes} & 1 \\
&4.5   &The Chain Rule; \textit{Chain Rule}  &   1.5 \\
&4.6   & Directional Derivatives and the Gradient; \textit{Partial Derivatives}   &1.5      \\
&4.7   &  Maxima/Minima Problems; \textit{Extrema}      &2        \\
&4.8   & \textbf{(Op.)} Lagrange multipliers               &  \\
&       &                                   &       \\
Chapter & 5 & \textbf{Multiple Integration} &            (total 9.5 hr) \\
&5.1 & Double Integrals over Rectangular Regions; \textit{Multiple Integrals} & 2 \\
&5.2 & Double integrals over General Regions; \textit{Multiple Integrals} & 2 \\
&5.3 & Double integrals in Polar Coordinates; \textit{Multiple Integrals} & 1.5 \\
&5.4 & Triple Integrals; \textit{Multiple Integrals}    &1.5 \\
&5.5 & Triple Integrals in Cylindrical and Spherical Coordinates; \textit{Multiple Integrals} & 2 \\
&5.6   & Calculating Centers of Mass and Moments of Inertia; \textit{Multiple Integrals}  &0.5 \\
&5.7   & \textbf{(Op.)} Change of Variables in Multiple Integrals&   \\
 &      &                                        &    \\
Chapter & 6    &\textbf{Vector Calculus}  & (total 14 hr) \\
 &6.1   & Vector Fields; \textit{Vector Fields}   & 1.5 \\
 &6.2 &  Line integrals; \textit{Line/Surface Integrals} & 2 \\
 &6.3   &Conservative Vector Fields; \textit{Vector Fields}  & 1.5 \\
 &6.4   & Green's Theorem; \textit{Important Theorems}    &1.5 \\
 &6.5   & Divergence and Curl;  \textit{Vector Fields}           &2 \\
 &6.6   & Surface Integrals; \textit{Line/Surface Integrals}  &2 \\
 &6.7   & Stokes' theorem; \textit{Important Theorems}                 &2 \\
 &6.8   & The Divergence Theorem; \textit{Important Theorems} & 1.5\\
 &       &                                       &       \\
 &    & Total Hours & 39
\end{tabular}



\end{document}
